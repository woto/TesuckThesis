\Introduction
Задача выделения ключевых слов и фраз из текста возникает в
библиотечном деле, лексикографии и терминоведении, а также в
информационном поиске. В настоящее время, объёмы и динамика
информации, которая подлежит обработке в этих областях делают
особенно актуальной задачу \emph{автоматического выделения}
ключевых слов и фраз. Выделенные таким образом слова и
словосочетания могут использоваться для создания и развития
терминологических ресурсов, а также для эффективной обработки
документов: индексирования, реферирования и классификации
\cite{Braslavsky06}.

Исследованию данной задачи посвящено множество работ
\cite{Medelyan09,Turney00,Frantzi00,Witten99}, предлагающих
различные методы выделения ключевых слов и фраз из текста,
построенные как на основе статистических, так и лингвистических
моделей.

Существует большое число доступных систем автоматического выделения
ключевых фраз, разработанных и ориентированных исключительно на
обработку западноевропейских языков
\cite{OpenCalais,Extractor,YahooTermExtractionWebService,TerMine,Maui}. Очевидно,
поддержка русского языка в таких системах отсутствует.

Системы автоматического извлечения ключевых фраз из текста на
естественном языке также разрабатывались в России
\cite{TextAnalyst,AOTSynAn,ContentAnalyzer,SemanticMirror},
но из-за особенностей их программной реализации,
применения устаревших и недостаточно эффективных методов извлечения
ключевых фраз \cite{Braslavsky06,Braslavsky07,Braslavsky08}
и условий распространения не могут быть использованы для решения
обозначенных выше прикладных задач.

Сегодня интеллектуальные информационные системы нашли широкое
применение в области здравоохранения. Одной из важнейших
составляющих современных медицинских информационных систем является
подсистема интеллектуального анализа текстовых данных, адекватность
функционирования которой напрямую зависит от качества работы модуля
автоматического извлечения ключевых фраз.

Таким образом, актуальны задачи исследования и разработки \emph{системы
автоматического извлечения ключевых фраз из текста на естественном
языке}.
