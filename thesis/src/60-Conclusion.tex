\Conclusion
В процессе построения системы автоматического извлечения
ключевых фраз из текста на естественном языке был проведён
обзор аналогов, выбран прототип, проведён его критический анализ,
указаны недостатки и пути их устранения. Это позволило сформулировать
цели и задачи для создания пакета моделей.

На этапе построения моделей получены содержательная, концептуальная,
структурная, функционально--структурная и алгоритмическая модели
предлагаемого решения.

Данные модели позволили уточнить функцию и структуру системы
автоматического извлечения ключевых фраз из текста на естественном
языке, найти пути устранения недостатков прототипа, а также
составить Техническое задание на программный продукт.

В ходе внутреннего проектирования решены задачи,
сформулированные в Техническом задании, исследована
структура и взаимосвязь составляющих системы автоматического
извлечения ключевых фраз из текста на естественном языке.

Реализован Tesuçk — система автоматического извлечения ключевых
фраз из текста на естественном языке. Web-интерфейс системы
доступ по адресу \url{http://tesuck.eveel.ru/}.

\section*{Дальнейшая работа}
Сегодня в ведущих иностранных научных сообществах
наблюдается большой интерес к системам синтеза изображения по
тексту (\emph{англ.} TTP — Text-to-Picture) и их приложениям
\cite{Goldberg09,Mihalcea08,Zhu07}. Использование таких систем
целесообразно, когда применение традиционного текстового
человеко--машинного интерфейса невозможно или недостаточно
эффективно \cite{Mihalcea08}.

Известно, что иллюстрация делает текст нагляднее и проще в
восприятии \cite{Zhu07}. Этот факт повсеместно используется
в различных дидактических системах, например при обучении
родному языку детей дошкольного возраста (или обучении
иностранном языку взрослых людей). Наличие иллюстраций в
тексте способствует пополнению словарного запаса обучающегося и
развитию его ассоциативного мышления \cite{Yoshii02}.

При помощи TTP-систем возможно автоматически синтезировать
изображения, отражающие ключевые фрагменты текста,
что позволяет широко применять их в компьютерных обучающих
системах \cite{Zhu07}.

Важнейшей задачей является медицинская реабилитация глухих людей,
или имеющих черепно--мозговые травмы, или различные умственные
нарушения \cite{Goldberg09} (например, синдром Дауна, и\ др.).
При помощи систем синтеза изображения по тексту возможно радикально
упростить представление учебной информации, тем самым обеспечив
врачей и педагогов качественным дидактическим материалом
\cite{Mihalcea08,Yoshii02}, сделав процесс обучения по-настоящему
наглядным.

Функционирование TTP-систем основано \cite{Zhu07} на
предварительном выделении \emph{ключевых слов и фраз} из
исходного текста, и последующей его обработкой при помощи методов
компьютерной лингвистики \cite{Manning99}, машинного обучения
\cite{Hastie08}, компьютерного зрения \cite{Forsyth02}, и\ др.

Несмотря на достоинства и возможности TTP-систем, не обнаружено
ни одной TTP-системы, способной работать с русскоязычными
текстами. Следовательно, целесообразно создание системы синтеза
изображения по тексту на основе современной системы автоматического
извлечения ключевых фраз.

На основе системы Tesuçk, созданной в рамках данной выпускной
квалификационной работы, \underline{возможно} построение современной
TTP-системы, способной обрабатывать русскоязычные тексты. 

\section*{Благодарности}
Автор благодарит C.\,И.~Кумкова за ценные замечания по содержанию
работы.
