\chapter{Техническое задание}
\label{chap:RequirementsSpecification}

\newpage
\thispagestyle{empty}
\begin{centering}

\addtolength{\parskip}{-0.25em}
{
\small
Министерство образования и науки Российской Федерации

Федеральное государственное автономное
образовательное учреждение

высшего профессионального образования

«Уральский федеральный университет имени первого
Президента России Б.\,Н.~Ельцина»
} \\
\vspace{0.3em}
Физико--технологический институт \\
\vspace{0.3em}
Кафедра вычислительной техники
\addtolength{\parskip}{0.25em}

\vspace{3em}

\begin{minipage}{.9\textwidth}
\begin{flushright}
  {\large{УТВЕРЖДАЮ:}\vspace{1ex}} \\
  Зав.\ кафедрой, д.\ т.\ н., профессор
  \vspace{0.5ex} \\
  \rule{30mm}{1pt}\hspace{3mm}С.\,Л.~Гольдштейн
  \vspace{0.5ex} \\
  «\rule{10mm}{1pt}» \rule{40mm}{1pt} 2011 г.
\end{flushright}
\end{minipage}

\vspace{4em}

{
  \large
  \textbf{СИСТЕМА АВТОМАТИЧЕСКОГО ИЗВЛЕЧЕНИЯ} \\
  \textbf{КЛЮЧЕВЫХ ФРАЗ ИЗ ТЕКСТА} \\
  \textbf{НА ЕСТЕСТВЕННОМ ЯЗЫКЕ} \\
  \vspace{1em}
  Техническое задание \\
}
на 16 листах \\
действует с «15» сентября 2010г.

\vspace{4em}

\begin{flushleft}
Согласовано: \\
\rule{3cm}{1pt} А.\,Г.~Кудрявцев \\
к.\ ф.-м.\ н., доц.\ каф.~ВТ УрФУ  \\
\end{flushleft}

\vfill

Екатеринбург \\
2011

\end{centering}

\newpage
\makeatletter
\def\reqspectableofcontents{\section*{\large\begin{center}Содержание\end{center}}\@starttoc{reqspectoc}}
\makeatother
\reqspectableofcontents
\newpage

\section*{1 Общие сведения}
\addcontentsline{reqspectoc}{section}{1 Общие сведения}
\subsection*{1.1 Полное наименование разработки и её условное обозначение}
\addcontentsline{reqspectoc}{subsection}{1.1 Полное наименование разработки и её условное обозначение}
Tesuçk — система автоматического извлечения ключевых фраз из текста на
естественном языке (далее — Система).

\subsection*{1.2 Шифр (номер) договора}
\addcontentsline{reqspectoc}{subsection}{1.2 Шифр (номер) договора}
Выпускная квалификационная работа по специальности 230200.62
«Информационные системы». 

\subsection*{1.3 Наименование предприятий разработчика и заказчика системы}
\addcontentsline{reqspectoc}{subsection}{1.3 Наименование предприятий разработчика и заказчика системы}
\begin{itemize}
  \item Разработчик — Усталов Дмитрий Алексеевич, студент группы
Фт-47081 кафедры «Вычислительная техника» физико--технологического
института ФГАОУ ВПО «УрФУ имени первого Президента России
Б.\,Н.~Ельцина».
  \item Заказчик — кафедра «Вычислительная техника»
физико--технологического института ФГАОУ ВПО «УрФУ имени первого
Президента России Б.\,Н.~Ельцина».
\end{itemize}

\subsection*{1.4 Перечень документов, на основании которых создаётся проект}
\addcontentsline{reqspectoc}{subsection}{1.4 Перечень документов, на основании которых создаётся проект}
Система создаётся на основе следующих документов:
\begin{itemize}
  \item материалы спецпрактикума;
  \item учебный план подготовки по специальности 230200.62.
\end{itemize}

\subsection*{1.5 Плановые сроки начала и окончания работы по созданию пакета}
\addcontentsline{reqspectoc}{subsection}{1.5 Плановые сроки начала и окончания работы по созданию пакета}
\begin{itemize}
  \item начало работ: 15.09.2010г.
  \item окончание работ: 15.05.2010г.
\end{itemize}

\subsection*{1.6 Порядок оформления и предъявления заказчику результатов работ по созданию системы}
\addcontentsline{reqspectoc}{subsection}{1.6 Порядок оформления и предъявления заказчику результатов работ по созданию системы}
\begin{itemize}
  \item в соответствии с ГОСТ~34.602 оформляется, согласуется и
утверждается Техническое задание. Заказчику передаётся Техническое
задание в виде отчёта;
  \item на основе Технического задания оформляется технорабочий
проект;
  \item по окончании этапа проектирования оформляется руководство
пользователя и разработчика;
  \item по окончании работ оформляется акт сдачи--приёмки выполненных
работ;
  \item оформляется пояснительная записка к дипломной работе.
\end{itemize}

\section*{2 Назначение и цели создания проекта}
\addcontentsline{reqspectoc}{section}{2 Назначение и цели создания проекта}
\subsection*{2.1 Назначение и перечень объектов автоматизации}
\addcontentsline{reqspectoc}{subsection}{2.1 Назначение и перечень объектов автоматизации}
Назначение: автоматизация процесса выделения многословных
терминов из текста на естественном языке.

Объект автоматизации: ручная обработка текста.

\subsection*{2.2 Цели разработки}
\addcontentsline{reqspectoc}{subsection}{2.2 Цели разработки}
Глобальная цель — автоматизация процесса выделения многословных
терминов из текста на естественном языке.

Локальные цели:
\begin{itemize}
  \item упростить организацию поиска информационных ресурсов
экспертом предметной области;
  \item ускорить процесс реферирования и агрегирования документов;
  \item обеспечить использование в качестве подсистемы
интеллектуального анализа данных.
\end{itemize}

\subsection*{2.3 Наименование и требуемые значения технических, экономических и социальных показателей, которые должны быть достигнуты}
\addcontentsline{reqspectoc}{section}{2.3 Наименование и требуемые значения технических, экономических и социальных показателей, которые должны быть достигнуты}
\subsubsection*{2.3.1 Технический показатель}
\addcontentsline{reqspectoc}{subsubsection}{2.3.1 Технический показатель}
Сокращение временных затрат на выделение ключевых слов и фраз из
текста на естественном языке в процессе индексации, аннотирования
и реферирования документов. Использование системы должно заметно
сокращать время, затрачиваемое экспертом предметной области при
анализе документов: компьютер лучше справляется с обработкой
больших объёмов информации.

\subsubsection*{2.3.2 Социальный показатель:}
\addcontentsline{reqspectoc}{subsubsection}{2.3.2 Социальный показатель}
\begin{itemize}
  \item экономия рабочего времени персонала;
  \item снижение трудоёмкости получения результата путём замены
ручного труда работой автоматической системы;
  \item повышение научного уровня за счёт возможности обработки
больших корпусов текстов за небольшой отрезок времени;
  \item повышение квалификации персонала за счёт использования в его
работе современных информационных технологий с простым и удобным
пользовательским интерфейсом.
\end{itemize}

\subsubsection*{2.3.3 Экономический показатель}
\addcontentsline{reqspectoc}{subsubsection}{2.3.3 Экономический показатель}
Снижение себестоимости эксперимента достигается в результате: 
\begin{itemize}
  \item замены ручного труда работой автоматической системы;
  \item экономия временных ресурсов;
  \item экономия трудовых ресурсов;
  \item экономия энергетических ресурсов.
\end{itemize}

\section*{3 Характеристика объектов автоматизации}
\addcontentsline{reqspectoc}{section}{3 Характеристика объектов автоматизации}
\subsection*{3.1 Краткие сведения об объекте автоматизации}
\addcontentsline{reqspectoc}{subsection}{3.1 Краткие сведения об объекте автоматизации}
Задача автоматизации — исследовать и реализовать систему автоматического
извлечения ключевых фраз из текста на естественном языке таким образом,
чтобы она могла работать в качестве интеллектуальной подсистемы системы
автоматической обработки документов.

Объект автоматизации — произвольный социоорганизационный.

\subsection*{3.2 Сведения об условиях эксплуатации объекта и характеристиках окружающей среды}
\addcontentsline{reqspectoc}{subsection}{3.2 Сведения об условиях эксплуатации объекта и характеристиках окружающей среды}
Оборудование, используемое при работе системы автоматического извлечения
ключевых фраз из текста на естественном языке, находится в
нормальных условиях окружающей среды. Серверное оборудование, содержащее
хранилище данных, функционирует круглосуточно, ежедневно, с перерывами на
обслуживание. Программное обеспечение, расположенное на рабочей станции
пользователя, даёт возможность пользования системой ежедневно и
круглосуточно.

\section*{4 Требования к системе}
\addcontentsline{reqspectoc}{section}{4 Требования к системе}
\subsection*{4.1 Требования к системе в целом}
\addcontentsline{reqspectoc}{subsection}{4.1 Требования к системе в целом}
Система должна выполнять следующие функции:
\begin{itemize}
  \item производить извлечение ключевых фраз из поступающих
текстов на естественном языке;
  \item предоставлять пользователю подробную информацию о
релевантности и значимости каждого потенциального термина в
проанализированном тексте при помощи Web-интерфейса;
  \item обеспечивать расширяемость путём предоставления
сотрудникам предприятия Заказчика собственного программного
интерфейса.
\end{itemize}

\subsubsection*{4.1.1 Требования к защите информации от несанкционированного доступа}
\addcontentsline{reqspectoc}{subsubsection}{4.1.1 Требования к защите информации от несанкционированного доступа}
Система не требует принятия каких--либо мер по защите
информации от несанкционированного доступа по причине
отсутствия необходимости хранения в базе данных промежуточных
вычислений, а также конечных результатов работы.

\subsubsection*{4.1.2 Требования к численности и квалификации персонала}
\addcontentsline{reqspectoc}{subsubsection}{4.1.2 Требования к численности и квалификации персонала}
К работе с Системой должны допускаться сотрудники, имеющие навыки
работы на персональном компьютере, ознакомленные с правилами
эксплуатации и прошедшие обучение по работе с Системой.

\subsubsection*{4.1.3 Требования к эргономике}
\addcontentsline{reqspectoc}{subsubsection}{4.1.3 Требования к эргономике}
Требования к эргономике заключаются в правильном построении
пользовательского интерфейса системы. Интерфейс разрабатываемого
пакета должен быть интуитивно понятным и доступным пользователю.
Цветовая гамма интерфейса не должна оказывать утомляющего действия
на пользователя.

\subsection*{4.2 Требования к видам обеспечения}
\addcontentsline{reqspectoc}{subsection}{4.2 Требования к видам обеспечения}
\subsubsection*{4.2.1 Математическое обеспечение}
\addcontentsline{reqspectoc}{subsubsection}{4.2.1 Математическое обеспечение}
Необходимо наличие функционально--структурных и алгоритмических
моделей. Математическая модель должна быть подробной, для того чтобы
можно было полностью осознать проблематику и методы решения
поставленных задач.

\subsubsection*{4.2.2 Информационное обеспечение}
\addcontentsline{reqspectoc}{subsubsection}{4.2.2 Информационное обеспечение}
Интерфейс пакета должен быть интуитивно понятным и доступным
пользователю, владеющему ПЭВМ на уровне пользователя. При
необходимости пользователь обращается к руководству пользователя. 

\subsubsection*{4.2.3 Лингвистическое обеспечение}
\addcontentsline{reqspectoc}{subsubsection}{4.2.3 Лингвистическое обеспечение}
Пакет должен быть реализован на платформе Rubinius по принципу
«тонкого клиента». Взаимодействие пользователя с системой
осуществляется через Web-интерфейс.

\subsubsection*{4.2.4 Программное обеспечение}
\addcontentsline{reqspectoc}{subsubsection}{4.2.4 Программное обеспечение}
При выборе программного обеспечения предпочтение должно отдаваться
архитектурным решениям и программным продуктам, уже доказавшим свою
пригодность при решении подобных задач. Базовое программное
обеспечение должно поддерживать и использовать стандартные сетевые
протоколы передачи данных.

Для создания Системы используется следующее программное
обеспечение:
\begin{itemize}
  \item операционная система Fedora Linux;
  \item язык программирования Ruby: современный динамический
объектный язык с элементами функционального программирования;
  \item программный каркас для разработки Web-приложений Sinatra;
  \item среда разработки: текстовый редактор Sublime Text 2 или
аналогичный;
  \item система управления исходным кодом: Git или любая другая
распределённая система контроля версий;
  \item система хранения пар «ключ--значение» Tokyo Cabinet;
  \item Web-сервер nginx;
  \item облачная PaaS-система Cloud Foundry.
\end{itemize}

Все перечисленные программные решения являются свободным программным
обеспечением, отлично зарекомендовавшим себя при использовании в
проектах различной сложности.

Пользователь должен иметь возможность воспользоваться Системой
при помощи любого популярного Web--браузера (Internet Explorer,
Mozilla Firefox, Google Chrome, Opera).

\subsubsection*{4.2.5 Техническое обеспечение}
\addcontentsline{reqspectoc}{subsubsection}{4.2.5 Техническое обеспечение}
В состав комплекса технических средств должны входить:
\begin{itemize}
  \item серверы баз данных;
  \item рабочие станции;
  \item сетевое оборудование;
  \item периферийное оборудование.
\end{itemize}

\subsubsection*{4.2.6 Организационное обеспечение}
\addcontentsline{reqspectoc}{subsubsection}{4.2.6 Организационное обеспечение}
Рекомендуется провести обучение пользователей работе с
пакетом (демонстрация).

\subsubsection*{4.2.7 Методическое обеспечение}
\addcontentsline{reqspectoc}{subsubsection}{4.2.7 Методическое обеспечение}
Необходимо наличие документа – руководства пользователя по работе с
системой автоматического извлечения ключевых фраз из текста на
естественном языке.

\section*{5 Состав и содержание работ по созданию работы}
\addcontentsline{reqspectoc}{section}{5 Состав и содержание работ по созданию работы}
\subsection*{5.1 Перечень стадий и этапов работ по созданию системы}
\addcontentsline{reqspectoc}{subsection}{5.1 Перечень стадий и этапов работ по созданию системы}
Следует ориентироваться на основные стадии (системное
проектирование, создание системно--обоснованного Технического
задания, эскизный проект, технорабочий проект) и этапы
(содержательное и концептуальное, полуформализованное и
математическое моделирование; а также конструкторско--технологическое
представление) проектирования.

Разработка системы включает в себя следующие этапы:
\begin{itemize}
  \item определение и анализ требований: определяется
нормативно--техническая документация, на основе которой создаётся
система, принимаются во внимание пожелания заказчика по функциям,
которые должна содержать разрабатываемая система;
  \item формирование спецификаций: на основе полученных данных о
системе, условий её эксплуатации и рекомендаций по улучшению
работы с ней создаётся системно--обоснованное Техническое задание;
  \item проектирование: разработка модели системы, определение
компонентов и их взаимосвязи между собой, определения перечня функций,
выполняемых каждым компонентом и системой в целом;
  \item кодирование: создание программного кода системы на основе
выбранного программного, лингвистического, математического и
информационного обеспечения;
  \item тестирование и верификация: проверка работоспособности
системы;
  \item документирование: подготовка необходимой документации
методического, технического и организационного характера по сдаче,
вводу в эксплуатацию данной системы;
  \item сопровождение: проведение различных мероприятий для обеспечения
работоспособности системы и её дальнейшему развитию.
\end{itemize}

\subsection*{5.2 Сроки выполнения этапов работ}
\addcontentsline{reqspectoc}{subsection}{5.2 Сроки выполнения этапов работ}
Состав и содержание работ на всех стадиях жизненного цикла
Системы представлены в таблицах~\ref{tab:Task1},
\ref{tab:Task2}, \ref{tab:Task3}, \ref{tab:Task4},
\ref{tab:Task5}, \ref{tab:Task6}, \ref{tab:Task7},
\ref{tab:Task8}.

\begin{table}[H]
\caption{\label{tab:Task1}
Формирование требований к Системе.}
\begin{center}
\begin{tabular}{|p{1cm}||p{9cm}|p{4cm}|}
\hline
№ & Этапы работы & Сроки выполнения \\
\hline
\hline
1 & Обследование объекта & 5 дней \\
\hline
2 & Формирование требований пользователя к Системе & 10 дней \\
\hline
3 & Оформление отчёта о выполненной работе и
заявки на разработку Системы & 10 дней \\
\hline
\end{tabular}
\end{center}
\end{table}
\begin{table}[H]
\caption{\label{tab:Task2}
Разработка концепции Системы.}
\begin{center}
\begin{tabular}{|p{1cm}||p{9cm}|p{4cm}|}
\hline
№ & Этапы работы & Сроки выполнения \\
\hline
\hline
1 & Изучение объекта & 5 дней \\
\hline
2 & Проведение необходимых научно--исследовательских
работ & 15 дней \\
\hline
3 & Разработка вариантов концепции Системы и выбор
концепции Системы, удовлетворяющего требования
пользователя & 10 дней \\
\hline
4 & Оформление отчёта о выполненной работе & 3 дня \\
\hline
\end{tabular}
\end{center}
\end{table}
\begin{table}[H]
\caption{\label{tab:Task3}
Техническое задание.}
\begin{center}
\begin{tabular}{|p{1cm}||p{9cm}|p{4cm}|}
\hline
№ & Этапы работы & Сроки выполнения \\
\hline
\hline
1 & Разработка и утверждение Технического задания на
создание Системы & 4 дня \\
\hline
\end{tabular}
\end{center}
\end{table}
\begin{table}[H]
\caption{\label{tab:Task4}
Эскизный проект.}
\begin{center}
\begin{tabular}{|p{1cm}||p{9cm}|p{4cm}|}
\hline
№ & Этапы работы & Сроки выполнения \\
\hline
\hline
1 & Разработка предварительных проектных решений по Системе
и её частям & 30 дней \\
\hline
2 & Разработка документации на Систему и её части & 14 дней \\
\hline
\end{tabular}
\end{center}
\end{table}
\begin{table}[H]
\caption{\label{tab:Task5}
Технический проект.}
\begin{center}
\begin{tabular}{|p{1cm}||p{9cm}|p{4cm}|}
\hline
№ & Этапы работы & Сроки выполнения \\
\hline
\hline
1 & Разработка проектных решений по Системе
и её частям & 30 дней \\
\hline
2 & Разработка документации на Систему и её части & 3 дня \\
\hline
3 & Разработка заданий на проектирование в смежных
частях проекта объекта автоматизации & 14 дней \\
\hline
\end{tabular}
\end{center}
\end{table}
\begin{table}[H]
\caption{\label{tab:Task6}
Рабочая докуметация.}
\begin{center}
\begin{tabular}{|p{1cm}||p{9cm}|p{4cm}|}
\hline
№ & Этапы работы & Сроки выполнения \\
\hline
\hline
1 & Разработка документации на Систему и её части & 7 дней \\
\hline
2 & Разработка или адаптация программы & 60 дней \\
\hline
\end{tabular}
\end{center}
\end{table}
\begin{table}[H]
\caption{\label{tab:Task7}
Ввод в эксплуатацию.}
\begin{center}
\begin{tabular}{|p{1cm}||p{9cm}|p{4cm}|}
\hline
№ & Этапы работы & Сроки выполнения \\
\hline
\hline
1 & Подготовка объекта автоматизации ко вводу Системы
в действие & 2 дня \\
\hline
2 & Подготовка персонала & 7 дней \\
\hline
3 & Пуско--наладочные работы & 2 дня \\
\hline
4 & Проведение предварительных испытаний & 4 дня \\
\hline
5 & Проведение опытной эксплуатации & 3 дня \\
\hline
6 & Проведение приёмочных испытаний & 2 дня \\
\hline
\end{tabular}
\end{center}
\end{table}
\begin{table}[H]
\caption{\label{tab:Task8}
Сопровождение Системы.}
\begin{center}
\begin{tabular}{|p{1cm}||p{9cm}|p{4cm}|}
\hline
№ & Этапы работы & Сроки выполнения \\
\hline
\hline
1 & Выполнение работ в соответствии с гарантийными
обязательствами & 7 дней \\
\hline
2 & Послегарантийное обслуживание & — \\
\hline
\end{tabular}
\end{center}
\end{table}

\section*{6 Порядок контроля приёмки системы}
\addcontentsline{reqspectoc}{section}{6 Порядок контроля приёмки системы}
\subsection*{6.1 Общие требования к приёмке работ}
\addcontentsline{reqspectoc}{subsection}{6.1 Общие требования к приёмке работ}
Состав, объём и методы испытаний должны быть определены в документах
«Программа и методика испытаний», являющимся неотъемлемой частью
соответствующей документации технического и рабочего проекта.

Приёмку работ должна осуществлять сформированная Заказчиком рабочая
группа из представителей Заказчика и Исполнителя.

Каждый выполненный Разработчиком этап принимается рабочей группой
Заказчика. При наличии положительного результата, окончательный приём
работы осуществляется Приёмочной комиссией.

Место проведения приёмочных испытаний должны быть согласованы Исполнителем
с Заказчиком на этапе технического и рабочего проектирования. Сроки
испытаний могут быть скорректированы Исполнителем и Заказчиком на этапе
технического и рабочего проектирования.

Все виды испытаний должны отвечать требованиям ГОСТ 34.603-92.

По результатам своей работы Приёмочная комиссия оформляет Акт приёмки
работ, который подписывается всеми членами Приёмочной комиссии и
представляется на утверждение Заказчику.

\subsection*{6.2 Виды испытаний}
\addcontentsline{reqspectoc}{subsection}{6.2 Виды испытаний}
\subsubsection*{6.2.1 Предварительные испытания}
\addcontentsline{reqspectoc}{subsubsection}{6.2.1 Предварительные испытания}
Предварительные автономные испытания специального программного обеспечения
Системы проводятся в соответствии с программой и методикой предварительных
автономных испытаний, подготовленной для каждого из функциональных модулей
Системы с использованием автономных тестов, подготовленных Разработчиком и
согласованных с Заказчиком.

Комплексный тест должен обеспечивать проверку выполнения функций
специального программного обеспечения (функциональных модулей),
установленных настоящим Техническим заданием, в том числе всех
связей между ними, а также проверку реакции подсистемы на некорректную
информацию и аварийные ситуации.

\subsubsection*{6.2.2 Опытная эксплуатация}
\addcontentsline{reqspectoc}{subsubsection}{6.2.2 Опытная эксплуатация}
Опытная эксплуатация проводится в реальном режиме работы организации
Заказчика. Все функции, автоматизация которых предусмотрена в Системе,
должны выполняться с использованием средств автоматизации.

\subsubsection*{6.2.3 Приёмочные испытания}
\addcontentsline{reqspectoc}{subsubsection}{6.2.3 Приёмочные испытания}
Приёмочные испытания проводятся в соответствии с Программой и методикой
приёмочных испытаний путём выполнения комплексных тестов, подготовленных
Разработчиком и согласованных с Заказчиком. 

\section*{7 Требования к составу и содержанию работ по подготовке объекта автоматизации ко вводу Системы}
\addcontentsline{reqspectoc}{section}{7 Требования к составу и содержанию работ по подготовке объекта автоматизации ко вводу Системы}
Для создания условий функционирования объекта автоматизации, при которых
гарантируется соответствие создаваемой информационной системой требованиям,
содержащимся в настоящем техническом задании, и возможность эффективного
использования Системы, в организации Заказчика на этапе работ
«Подготовка объекта автоматизации к вводу системы в действие» должен
быть проведён следующий комплекс мероприятий:
\begin{itemize}
  \item установка и модернизация оборудования;
  \item обучение персонала.
\end{itemize}

\subsection*{7.1 Требования к документированию}
\addcontentsline{reqspectoc}{subsection}{7.1 Требования к документированию}
Документы, создаваемые в процессе работ по созданию
информационной системы, должны разрабатываться в
соответствии с ГОСТ 34.602-89 и РД 50-34.698-90.

Перечень документов, подлежащих разработке в рамках
создания Системы:
\begin{itemize}
  \item «Программа и методика испытаний», ГОСТ 34.602-89;
  \item «Руководство пользователя», РД 50-34.698-90.
\end{itemize}

\newpage
\section*{\large\begin{center}Источники разработки\end{center}}
\addcontentsline{reqspectoc}{section}{Источники разработки}
\begin{itemize}
  \item ГОСТ 34.003-90 Автоматизированные системы. Термины и определения.
— Введ. 01.01.90. — М.: Издательство стандартов, 1989.
  \item ГОСТ 34.601-90 Автоматизированные системы. Стадии создания.
— Введ. 01.01.90. — М.: Издательство стандартов, 1989.
  \item ГОСТ 34.602-89 Комплекс стандартов на автоматизированные системы.
Техническое задание на создание автоматизированной системы.
— Введ. 24.03.89. — М.: Издательство стандартов, 1989.
  \item ГОСТ 19.102-77 Единая система программной документации.
— Введ. 01.01.90. — М.: Издательство стандартов, 1989.
  \item ГОСТ 34.603—92 Виды испытаний автоматизированных систем.
— Введ. 01.01.93. — М.: Издательство стандартов, 1989.
  \item РД 50-34.698-90. Автоматизированные системы. Требования к
содержанию документов.
\end{itemize}
