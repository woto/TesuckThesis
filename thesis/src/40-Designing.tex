\chapter{Проектирование предлагаемого решения}
\label{chap:Designing}

\section{Внешнее проектирование}
В рамках проведения внешнего проектирования разработано Техническое
задание, которое приведено в приложении
\ref{chap:RequirementsSpecification}.

\section{Внутреннее проектирование}
Внутреннее проектирование проводилость на основании моделей,
разработанных и описанных в главе \ref{chap:Modeling}. В ходе
внутреннего проектирования решались задачи, сформулированные
в Техническом задании.

Для создания системы автоматического извлечения ключевых фраз из текста
на естественном языке была выбрана концепция мультипарадигменного
программирования, в частности объектно--ориентированного и
функционального программирования.

В качестве основной платформы для разработки выбран Rubinius —
перспективная реализация языка программирования Ruby \cite{Ruby}
на основе виртуальной машины LLVM \cite{Rubinius}. В соответствии с
современными требованиями и тенденциями, предлагаемое решение
построено на основе архитектуры REST \cite{Fielding00}, а также
стандартных протоколов: JSON \cite{JSON} и XML \cite{XML}.

\section{Результаты и выводы}
В ходе выполнения внешнего и внутреннего проектирования системы
автоматического извлечения ключевых фраз из текста на естественном языке
получены следующие результаты:
\begin{itemize}
  \item разработано Техническое задание на создание системы;
  \item определён состав инструментальных средств для создания
системы.
\end{itemize}

На основе результатов проектирования можно сделать следующий
вывод: разработанный проект системы соответствует Техническому
заданию.
